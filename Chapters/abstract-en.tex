%!TEX root = ../template.tex
%%%%%%%%%%%%%%%%%%%%%%%%%%%%%%%%%%%%%%%%%%%%%%%%%%%%%%%%%%%%%%%%%%%%
%% abstract-en.tex
%% NOVA thesis document file
%%
%% Abstract in English
%%%%%%%%%%%%%%%%%%%%%%%%%%%%%%%%%%%%%%%%%%%%%%%%%%%%%%%%%%%%%%%%%%%%

\typeout{NT FILE abstract-en.tex}%

This thesis presents the design and implementation of finite state transducers (FSTs) within the OCamlFLAT/OFLAT application, 
a web-based educational tool for exploring concepts in computation theory developed at FCT NOVA. 
Finite state transducers extend finite automata by associating output symbols with transitions, 
enabling the modeling of transformations from input to output strings. 

The current system consists of two separate components:
Firstly we have the OCamlFLAT library, a library with all the data structures and logic regarding the theoretical concepts of its supported models.
And secondly the OFLAT web application, which is built on top of the OCamlFLAT library, provides a pedagogical interactive graphic interface 
for users to explore the models supported by the library.
Both tools were written in OCaml, deliberately aiming to use a functional declarative style, 
which in some cases manages to closely resemble the theoretical formalisms taught to students.
It is a system that is in constant evolution, with new features being added to both the library and the web application.

The work to be developed in this thesis involves the extension of these tools with new operations over transducers, 
such as translation, acceptance, generation, classification and conversion.
Additionally, the new concepts must be integrated in OFLAT's ecosystem, which includes model composition,
and pedagogical exercises for students to practice and test their knowledge.

The resulting system will allow users to build, modify, and analyze FSTs visually, 
supporting better comprehension of theoretical principles through experimentation.
