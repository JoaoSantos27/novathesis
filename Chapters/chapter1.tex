%!TEX root = ../template.tex
%%%%%%%%%%%%%%%%%%%%%%%%%%%%%%%%%%%%%%%%%%%%%%%%%%%%%%%%%%%%%%%%%%%
%% chapter1.tex
%% NOVA thesis document file
%%
%% Chapter with introduction
%%%%%%%%%%%%%%%%%%%%%%%%%%%%%%%%%%%%%%%%%%%%%%%%%%%%%%%%%%%%%%%%%%%

\typeout{NT FILE chapter1.tex}%

\chapter{Introduction}
\label{cha:introduction}

\section{Context}

Theory of Computation is a fundamental area in computer science that includes various
fields of study, including this work's focus: Formal Languages and Automata
Theory (FLAT). These play a critical role in modeling computational processes, verifying system behavior, 
and designing compilers and interpreters. 

Despite its theoretical significance, FLAT concepts can often be difficult to understand due to their abstract nature. 
Especially for students new to the field, to address this, educational
tools for visualization and simulation of these concepts are an invaluable resource. 
Among these tools, OCamlFLAT/OFLAT, a system developed at NOVA School of Science and Technology,
provides a comprehensive platform for exploring various models of computation, including finite automata, grammars, pushdown automata, and Turing machines.
However despite its extensive support for various models, OCamlFLAT/OFLAT currently lacks support for finite-state transducers. 

Finite-State Transducers (FSTs) are an extension of the concept of finite automata, 
as they introduce output generation in response to input processing.
And while FSTs are well-established in theoretical literature, pedagogical tools 
that allow for their interactive construction, simulation, and analysis remain scarce or limited in functionality.

This dissertation aims to address this issue by extending both the library and the web interface to support the representation, manipulation, and simulation of FSTs. 
The goal being that this addition will make OCamlFLAT/OFLAT an even more complete educational tool for students and educators alike.

\section{Objectives}

The primary goal of this thesis is to design and implement support for finite-state transducers within the OCamlFLAT/OFLAT ecosystem. This includes:
\begin{itemize}
  \item Defining appropriate data structures and logic to model FSTs.
  \item Implementing key operations such as word translation, language recognition, determinization, and minimization.
  \item Integrating FSTs into the OFLAT graphical interface with interactive features for construction, simulation, and step-by-step word processing.
  \item Ensuring consistency with existing models, including support for composition and pedagogical exercises.
\end{itemize}

By enriching the system with transducer functionality, we aim to provide a more comprehensive pedagogical platform that supports a broader spectrum of the 
Chomsky hierarchy and theoretical computation models. This will enhance both teaching and learning by enabling students and instructors to explore complex language 
transformations through hands-on experimentation.

\section{Structure of the Document}

The remainder of this document is structured as follows. 
\begin{itemize}
  \item Chapter~\ref{cha:background} presents the theoretical foundations of formal languages and automata theory, focusing on finite-state transducers and their operations. 
  \item Chapter~\ref{cha:related_work} reviews existing related tools for simulation of various models and compares them to the OCamlFLAT/OFLAT system. 
  \item Chapter~\ref{cha:current_system} delves into the current functionalities and user experience of OCamlFLAT/OFLAT. 
  \item Chapter~\ref{cha:technologies} describes the technologies and frameworks used in the development of the OCamlFLAT/OFLAT system, including the OCaml programming language and the OFLAT web interface
  \item Chapter~\ref{cha:work_plan} outlines the detailed work plan for extending the system with FST support.
\end{itemize}

