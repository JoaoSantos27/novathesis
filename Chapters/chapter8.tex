%!TEX root = ../template.tex
%%%%%%%%%%%%%%%%%%%%%%%%%%%%%%%%%%%%%%%%%%%%%%%%%%%%%%%%%%%%%%%%%%%%
%% chapter5.tex
%% NOVA thesis document file
%%
%% Chapter with lots of dummy text
%%%%%%%%%%%%%%%%%%%%%%%%%%%%%%%%%%%%%%%%%%%%%%%%%%%%%%%%%%%%%%%%%%%%

\typeout{NT FILE chapter8.tex}%

\chapter{Work plan}
\label{cha:work_plan}

This chapter presents the work plan for this dissertation, considering a period of six months.
The objective of this work is to add support for finite state transducers (FSTs) to the OCamlFLAT/OFLAT educational system, 
considering three particular cases: the general case, Mealy machines, and Moore machines. A finite state transducer is a finite automaton that, 
in addition to recognizing words, can produce output during the recognition process. Therefore, an FST also translates an input language into an output language.

The models currently supported in OCamlFLAT/OFLAT are: finite automata, regular expressions, grammars 
(context-free, context-sensitive, and unrestricted), pushdown automata, and Turing machines.

\section{Support for Finite State Transducers in the OCamlFLAT Library (8 weeks)}

Several functionalities are planned at the library level. Some features might appear to directly extend those of finite automata, 
but the reality is more complex, for example, not all nondeterministic transducers can be determinized.

The planned functionalities include:

\begin{itemize}
    \item \textbf{Cleaning function} -- Eliminate from a transducer all unreachable and unproductive states.

    \item \textbf{Classification functions} -- Classify a transducer as nondeterministic, deterministic, determinizable, minimal, Mealy machine, Moore machine, cleaned, etc.

    \item \textbf{Accept function} -- Test if a transducer recognizes a given word and generates the corresponding output word. This will involve studying and instantiating the generic \texttt{accept} operation already available in the library. For words recognized by the transducer, produce a demonstration in the form of a sequence of sentential rewrites.

    \item \textbf{Generate function} -- Determine all words recognized by a transducer up to a given maximum length, along with their corresponding translations. This will involve studying and instantiating the generic \texttt{generate} operation available in the library.

    \item \textbf{Conversion functions} -- Convert a transducer to other model types where it makes sense; for example, convert a transducer to a finite automaton (removing the output), convert a finite automaton to a transducer (using the ``identity'' translation), or convert a transducer to a Turing machine (since Turing machines can represent output).

    \item \textbf{Integration with compound models} -- Integrate transducers into the existing support for compound models in the library. This requires studying composition operations between transducers: union, concatenation, intersection, Kleene closure, etc.

    \item \textbf{Integration with pedagogical exercises} -- Include transducers in the existing support for exercises in the OCamlFLAT library.
\end{itemize}

\section{Support for Finite State Transducers in the OFLAT Web Application (6 weeks)}

On the graphical side, the existing support for other models---particularly finite automata, which are closely related to transducers---is helpful. However, fully understanding the current implementation is non-trivial, and there are new aspects to handle, related to the fact that transducers generate output.

The tasks include:

\begin{itemize}
    \item \textbf{Design and implement a graphical representation for transducers} -- Based on graphs and tables, the two representations most commonly used in textbooks. The essential operations will be visualization and editing.

    \item \textbf{Expose transducer operations in the graphical interface} -- Make available in the web interface the operations on transducers implemented in the OCamlFLAT library.

    \item \textbf{Graphically animate word translation} -- Animate the process of translating a word through a transducer.

    \item \textbf{Maintain consistency with other models} -- In general, the functionalities should be consistent with those developed for the other models in OFLAT.
\end{itemize}

\section{Pedagogical Exercises (1 week)}

Create a collection of interesting pedagogical examples and encode them as a set of exercises in the format supported by OCamlFLAT/OFLAT.

\section{Evaluation (1 week)}

\begin{itemize}
    \item \textbf{Code correctness} -- Evaluate the correctness of the new code by writing and executing a suite of unit tests during the development process.

    \item \textbf{Stress testing} -- Determine the system’s limits by creating examples that challenge it in terms of size or execution time.

    \item \textbf{User feedback} -- Seek opinions from potential users of the system, especially regarding the graphical interface. For example, consulting instructors and students from the Computation Theory course.
\end{itemize}

\section{Writing the Dissertation (8 weeks)}

The dissertation writing should ideally progress incrementally in parallel with the other tasks. A period of two months is allocated for producing the final document. It is advisable to have a near-final version ready at least two weeks before the submission deadline to facilitate inviting an external examiner.




% \lipsum[1-100]
% \lipsum[1-700]
% \lipsum[1-700]
% \lipsum[1-700]
% \lipsum[1-700]
% \lipsum[1-700]
% \lipsum[1-700]
% \lipsum[1-700]
% \lipsum[1-700]
