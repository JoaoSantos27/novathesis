%!TEX root = ../template.tex
%%%%%%%%%%%%%%%%%%%%%%%%%%%%%%%%%%%%%%%%%%%%%%%%%%%%%%%%%%%%%%%%%%%%
%% chapter4.tex
%% NOVA thesis document file
%%
%% Chapter with lots of dummy text
%%%%%%%%%%%%%%%%%%%%%%%%%%%%%%%%%%%%%%%%%%%%%%%%%%%%%%%%%%%%%%%%%%%%

\typeout{NT FILE chapter4.tex}%

\chapter{Technologies}
\label{cha:technologies}

This chapter provides an overview of the key technologies used in the system. We present
the OCaml programming language, the mechanisms for interoperability between OCaml and JavaScript, 
and the use of additional libraries (such as Cytoscape.js) that enable both the functional and graphical aspects of
the platform.

\section{OCaml Programming Language}

OCaml is a general-purpose, high-level, multi-paradigm programming language that
extends the Caml dialect of ML with object-oriented features. It was created in 1996 by
Xavier Leroy, Jérôme Vouillon, Damien Doligez, Didier Rémy, Ascánder Suárez, and others
at the French Institute for Research in Computer Science and Automation (INRIA). The
functional programming fragment of OCaml is based on the concept of expressions, where
computations are performed by evaluating expressions that produce values, naturally
supporting a declarative programming paradigm. \cite{ocaml_about}

\subsection{Declarative Programming}

Declarative programming focuses on writing code that solves problems by describing their logical properties rather than by elaborating an algorithm filled with operational details. 
In declarative programming\cite{declarative-programming}, the emphasis is on specifying what the program should accomplish, rather than how to achieve it step by step.

For example, to determine the length of a list, a declarative solution might consist of a simple statement of the property: “the length of a list is one plus the length of its tail.” 
In contrast, an imperative solution would typically involve initializing a counter to zero and then traversing the list, incrementing the counter for each element.
Declarative programming is often considered very high-level because it is closer to mathematical methods and human thought processes, 
as opposed to programming in a way that mimics machine operations.

Recursive functions written in a declarative style are always inductive functions, in which the problem is reduced to smaller instances of the same problem. 
This makes many complex problems easier to solve. The code also becomes more readable, as it resembles an equation and does not require mentally simulating its execution to understand it. 
When writing inductive functions, it greatly helps if the data themselves are inductive in nature—for example, using lists or trees.
There are situations where writing a declarative solution still requires sophisticated considerations. 
For instance, to search for a value in an infinite binary tree, one must integrate the need to perform a breadth-first traversal into the solution. 
The exploration of infinite data structures is frequent in this project, as the configuration spaces that need to be explored are often infinite.


\section{Interoperability using js\_of\_ocaml}

One of the fundamental technologies enabling the OCamlFLAT/OFLAT system to function as a web-based tool is the \texttt{js\_of\_ocaml} framework. 
js\_of\_ocaml\cite{js-of-ocaml-manual} is the interoperability solution used for bridging OCaml and JavaScript. 
Enabling the use of OCaml's type system and functional programming features in web development.
With this we can code the entire tool in OCaml.
js\_of\_ocaml is not only a compiler but a collection of tools that includes a library, a binding mechanism, and a syntactic extension.

\section*{Compiler}

At the core of \texttt{js\_of\_ocaml} is its compiler, which translates OCaml code into JavaScript code. 
This compilation step typically follows the standard OCaml compilation pipeline: 
the source code is first compiled using \texttt{ocamlc}, and the result is then passed to \texttt{js\_of\_ocaml} for translation into JavaScript.
This allows us to have the type safety and early error detection provided by OCaml.

\subsection{Bindings}
Bindings is a crucial component for enabling interoperability between OCaml and JavaScript. 
In the \texttt{js\_of\_ocaml} ecosystem, bindings are used extensively to connect both languages, 
allowing OCaml code to access and manipulate JavaScript variables, functions and prototypes.

For example, the JavaScript alert function can be made available from OCaml using the following binding:
\begin{verbatim}
  let alert (msg: string): unit = Js.Unsafe.global##alert (Js.string msg)
\end{verbatim}

This snippet wraps the browser's window.alert in an OCaml function, converting the
OCaml string to a JavaScript string before invocation.

As a second example, the following definition makes available on the OCaml side, the JavaScrip constructor for the Cytoscape objects (that represents graphs).

\begin{verbatim}
  let cytoscape_constructor: Js.Unsafe.t = Js.Unsafe.pure_js_expr "Cytoscape"
\end{verbatim}

The following OCaml type, as shown in Listing~\ref{lst:cytoscape_binding}, describes the structure of the Cytoscape objects, and it allows method invocation with type safety:

\begin{lstlisting}[caption={Binding for Cytoscape object}, label={lst:cytoscape_binding}]
class type cytoscape =
object
  method add : DataItem.t t -> unit meth
  method remove : DataItem.t t -> unit meth
  method remove_fromSelector : js_string t -> unit meth
  method mount : Dom_html.element t -> unit meth
  method layout : layout_options t -> layout t meth
  method resize : unit meth
  (*...*)
  method edges : js_string t -> DataItem.t Js.t js_array Js.t meth
  method nodes : js_string t -> DataItem.t Js.t js_array Js.t meth
  method tapdragover : DataItem.t t -> unit meth
  method contextMenus : 'c t -> contextMenus Js.t meth
  method popper : 'z -> popper Js.t meth
end
\end{lstlisting}

\section*{Syntax Extensions}

To make JavaScript interoperation more idiomatic in OCaml, 
\texttt{js\_of\_ocaml} provides syntax extensions that support type-safe method invocation and property access. 
These extensions enable a natural coding style that mimics JavaScript syntax while remaining within OCaml's type system.

\section*{Library}

Beyond the compiler and bindings, \texttt{js\_of\_ocaml} includes a library that exposes a substantial portion of the browser's API. 
This includes modules for:
\begin{itemize}
  \item DOM manipulation (\texttt{Dom\_html}, \texttt{Dom})
  \item Event handling
  \item HTML5 APIs
  \item Canvas and graphics operations
\end{itemize}


\section{Cytoscape.js for Graph Visualization}

Cytoscape.js is an open-source JavaScript library for graph theory analysis and visualization~\cite{cytoscape}. 
In the context of the OFLAT system, it is used to render automata and tree structures graphically, 
providing users with an intuitive and interactive interface to construct and manipulate formal models. 

To integrate Cytoscape.js with OCaml, the OFLAT system uses bindings defined in the \texttt{Cytoscape.ml} module. 
Graph operations such as adding nodes and edges are mapped to JavaScript via \texttt{js\_of\_ocaml}.

\section{Web Technologies Used in OFLAT}

Understanding OFLAT's web-based architecture requires familiarity with several core technologies:
\begin{itemize}
    \item \textbf{Web browsers} act as clients that fetch and render web content, translating HTML, CSS, and JavaScript into interactive interfaces~\cite{webbrowsers}.
    \item \textbf{HTML} provides the structural markup for web pages, defining elements like headers, inputs, and buttons~\cite{html}.
    \item \textbf{DOM} (Document Object Model) represents the HTML document as a tree of nodes that can be dynamically modified via JavaScript~\cite{dom}.
    \item \textbf{JavaScript} enables dynamic behavior and interaction within web pages, and is essential for manipulating the DOM~\cite{javascript}.
\end{itemize}


