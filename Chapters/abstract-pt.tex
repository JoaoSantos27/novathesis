%!TEX root = ../template.tex
%%%%%%%%%%%%%%%%%%%%%%%%%%%%%%%%%%%%%%%%%%%%%%%%%%%%%%%%%%%%%%%%%%%%
%% abstract-pt.tex
%% NOVA thesis document file
%%
%% Abstract in Portuguese
%%%%%%%%%%%%%%%%%%%%%%%%%%%%%%%%%%%%%%%%%%%%%%%%%%%%%%%%%%%%%%%%%%%%

\typeout{NT FILE abstract-pt.tex}%

Esta dissertação apresenta o projeto e a implementação de transdutores finitos (FSTs) na aplicação OCamlFLAT/OFLAT, 
uma ferramenta educacional baseada na web para a exploração de conceitos de teoria da computação, 
desenvolvida na FCT NOVA. Os transdutores finitos estendem os autómatos finitos ao associar símbolos de saída às transições, 
permitindo a modelação de transformações entre cadeias de entrada e de saída.

O sistema atual é composto por dois componentes distintos: a biblioteca OCamlFLAT, 
que contém todas as estruturas de dados e lógica associadas aos conceitos teóricos dos modelos suportados; 
e a aplicação web OFLAT, construída sobre a biblioteca OCamlFLAT, que fornece uma interface gráfica interativa com fins pedagógicos 
para que os utilizadores possam explorar os modelos suportados pela biblioteca. Ambas as ferramentas foram desenvolvidas em OCaml, 
procurando adotar deliberadamente um estilo funcional declarativo, que em alguns casos se aproxima dos formalismos teóricos ensinados aos alunos. 
Trata-se de um sistema em constante evolução, com novas funcionalidades a serem acrescentadas continuamente tanto à biblioteca como à aplicação web.

O trabalho a desenvolver nesta dissertação envolve a extensão destas ferramentas com novas operações sobre transdutores, 
tais como tradução, aceitação, geração, classificação e conversão. 
Adicionalmente, os novos conceitos devem ser integrados no ecossistema do OFLAT, 
o que inclui a composição de modelos e a criação de exercícios pedagógicos para os alunos praticarem e testarem os seus conhecimentos.

O sistema resultante permitirá aos utilizadores construir, modificar e analisar transdutores finitos de forma visual, 
promovendo uma melhor compreensão dos princípios teóricos através da experimentação.
